\documentclass[conference]{IEEEtran}

\usepackage{cite}

\usepackage[cmex10]{amsmath}

\usepackage[pdftex]{graphicx}
\usepackage{setspace}
\usepackage{tabularx}
\usepackage{amsfonts,amssymb}

\usepackage{algorithm}
\usepackage[noend]{algpseudocode}

\usepackage[tight,footnotesize]{subfigure}



\begin{document}

\newtheorem{example}{\textbf{Example}}[section]
\newtheorem{definition}{\textbf{Definition}}
\newtheorem{lemma}{\textbf{Lemma}}
\newtheorem{theorem}{\textbf{Theorem}}
\newtheorem{claim}{\textbf{Claim}}
\newtheorem{corollary}{\textbf{Corollary}}
\newtheorem{observation}{\textbf{Observation}}
\newtheorem{property}{\textbf{Property}}



\title{How to write papers using LaTex}

\author{\IEEEauthorblockN{
Chick\IEEEauthorrefmark{1},
Duck\IEEEauthorrefmark{2}}
\IEEEauthorblockA{\IEEEauthorrefmark{1}State Key Laboratory for Novel Software Technology, Nanjing University, Nanjing 210024, China}
\IEEEauthorblockA{\IEEEauthorrefmark{2}Department of Computer Science, City University of Hong Kong, Hong Kong}
}

\maketitle

\begin{abstract}
This article describes how to use LaTex to write academic papers. It is written to help absolute beginners to gain a glimpse of how academic paper is organized using LaTex. Technical details and fancy tricks of LaTex will not be covered in this article(as I do not know any of them). Hopefully, this can serve as a template or maybe a reference when some of us set out to write a paper. 
\end{abstract}

% It is common practice to organize your paper into multiple latex files
% Here, we put only abstract in the main.tex

%!TEX root = main.tex
\section{Introduction}
\label{sec:intro}
In Introduction, we introduce the background of our work, describe briefly the problems we discover and the contribution we make.
%!TEX root = main.tex
\section{Related Work}
\label{sec:related}

In this section, we introduce related prior work regarding this paper's research topic. Usually, this section involves lots of citations.
citation works like this: \cite{adams_davies_2009}.
%!TEX root = main.tex
\section{Problem and Statement}
\label{sec:model}

BGMDD(bipartite graph matching with dynamic duration) problem(Sec. \uppercase\expandafter{\romannumeral3}-A)  and important concepts such as regret(Sec. \uppercase\expandafter{\romannumeral3}-B) are defined in this section. Table \uppercase\expandafter{\romannumeral1} lists the notation and basic definitions.

\subsection{Problem Definition}

\begin{definition}
(Dynamic Bipartite Graph, DBG). \emph{A dynamic bipartite graph is defined as $B=(L,R,E)$, where $L=\{i \in \mathbb{N}\}$ and  $R={j \in \mathbb{N}}$ are  the sets of left and right nodes and $E\in L\times R$ is the set of edges between $L$ and $R$. The node in $L$ or $R$ arrives independently from known probability distributions $P_{L}=\{p_l\}$ or $P_{R}=\{p_r\}$. Each node $i(j)$ arrives at time i(j), which is meaning to abuse the index to denote the nodes' arriving time. Each node has duration denoted by $i.d(j.d)$. If a node is not matched during the duration, the node would leave. Each edge $(i,j)\in E$ has a weight denoted by $e_ij$ obeyed a distribution $P_E=\{p_e\}$.}
\end{definition}
\begin{definition}
(Matching Allocation). \emph{A matching allocation over a dynamic bipartite graph B is denoted by $M =\{(i, j)|i \in L, j \in R\}$. It is a set of node pairs where each node appears at most once.The utility score of a matching allocation $M$ over a dynamic bipartite graph $B$ is measured by $U(B, M) = \{(i,j)\in M w(i, j)\}$.}
\end{definition}

\par In a real word, the nodes arriving distributions ($P_{L}=\{p_l\}$  $P_{R}=\{p_r\}$ ) and the edge weight distribution($P_E=\{p_e\}$) may change after some time. But it's easy to confirm the changes by collecting statistics. So in order to analysis convenient, we assume the distributions($P_{L}=\{p_l\}$, $P_{R}=\{p_r\}$ and $P_E=\{p_e\}$) are permanent. Alough by collecting statistics some changes could be measured, the change of duration of nodes can not be measured unless we doesn't match nodes and wait until the nodes leaving. It's impossible to do that to measure the changed of the duration. 

\begin{definition}
(Dynamic Duration Distribution.) \emph{Each node's duration obey a distribution $P_ld$(or $P_rd$) independently. In this paper, we first analyze the situation that the durations of nodes in  $L$ and $R$ obey the same distribution $P_d$, $P_d=P_ld= P_rd$. $P_d$ can change after some time unknown.  And we assume the type of the distribution should be pernament. The change of $P_d$ is the decrease or increase of expectation of distribution.  The length of the time in which $P_d$ is stable should not be short. }
\end{definition}
\par The change of duration distribution could be observed in many application scenarios. For example, in the food felivered scene, the customers' patient (which could be considered as duration) are always good in the early time,like 10:30 am (people are not very hungry), but very bad in1:30 am. The same situation would exist in online taxi-hailing service.
\begin{example}
xxxxxxxx
\end{example}
\begin{definition}
(BGMDD problem). \emph{Give a dynamic bipartite graph with duration changed dynamic, the BGMDD problem is to find a matching allcation $M$ to maximize the utility score in the online scenario. }
\end{definition}
\par The BGMDD problem inherits and develops the two-sided online maximum bipartite matching problem in which the duration is  given upon nodes' arrival. And it's also different from ......xxxx

\subsection{ Evaluation Metric}
\begin{property}
Kittens are cute.
\end{property}

\begin{lemma}
People love cute things.
\end{lemma}

\begin{theorem}\label{theorem:TheFatTreeInApx}
People love kittens.
\end{theorem}

\begin{IEEEproof}
Trivial.
\end{IEEEproof}

\subsection{LP formulation}
An example:

\begin{subequations}\label{opt:mcup}
\begin{align}
\text{minimize} & \quad\max_{e \in E,s \in \{1, 2, \dots, n-1 \}} \mu_e^s \quad\quad\quad\quad\quad\quad\quad\quad\quad\quad\eqref{opt:mcup}\notag\\
\text{subject to} & \quad {\sum_{f \in F_{sp} \cup F_{mp}} d^f {\sum_{p \in P(f): e\in p} \max(x_{f,p}^s},x_{f,p}^{s+1})}
\leq \mu_e^s C_{e}, \notag \\
& \quad\quad\forall e \in E, \forall s \in \{1, 2, \dots, n-1 \}, \label{equ:CapacityConstraint}\\
& \quad{\sum_{p \in P(f)} x_{f,p}^s} = 1, \notag \\
& \quad\quad\forall f \in F_{sp} \cup F_{mp}, \forall s \in \{2, 3, \dots, n-1 \}, \label{equ:SumConstraint}\\
& \quad x_{f,p}^s \in \{0,1\}, \notag \\
& \quad\quad\forall f \in F_{sp}, \forall p \in P(f), \forall s \in \{2, 3, \dots, n-1 \}, \label{equ:SinglePath}\\
& \quad x_{f,p}^s \geq 0, \notag \\
& \quad\quad\forall f \in F_{mp}, \forall p \in P(f), \forall s \in \{2, 3, \dots, n-1 \},
\label{equ:Multipath}\\
& \quad\mu_e^s > 0, \forall e \in E, \forall s \in \{1, 2, \dots, n-1 \}. \label{equ:ObjPositiveCostraint}
\end{align}
\end{subequations}

%!TEX root = main.tex
\section{Solution based on bandit with static-batch arms }
\label{sec:alg}
\par In this section, we would describe the basic idea why we need the bandit algorithm to solve BGMDD problem and the detail about the algorithm based on bandit with static-batch arms. At first, a approach to turn $P_d$ into a const would be introduced to make the following analysis easier.  Then, the batch way to slove DBG probelm would be introduced and the bound of the score expectation using a specific static batch would be calculated. Next, we would state that it's a bandit problem when we use batch way to solve BGMDD problem and the bandit algorithm we use in this paper, discounted UCB, would be introduced. Finally, the theoretical performance of above algorithms in BGMDD problem would be anaylze.
\subsection{Expectation Approximate}
\par As the statement in section xx, the durations of nodes always obey some distribution in real application scenarios. But it's difficult to anaylze the performance of alforithms when the duration is a distribution. So we abuse the expectation of the distribution, $E_d$, to replace $P_d$ in the following anaylsis. When the duration is a constant,i.e. $E_d$, we use $T_e$ to represent the total score. When the duration is a distribution, i.e. $P_d$, we use $T_p$ to represent the total score. Then we can get the difference when we use $E_d$ to represent $P_d$ in the anaylsis in \textbf{Theorem 1}.
\begin{theorem}\label{theorem:TheFatTreeInApx}
When the distribution is limited(like 99\% value in distribution is limited), we can find the max value and min value in $P_d$, denoted by $D_{max}$ and $D_{min}$. When duration is $D_{max}$ or $D_{min}$, the total score is represented by $T_{D_{max}}$ or $T_{D_{min}}$. The difference between $T_d$ and $T_e$ can be no higher then the difference between $T_{D_{max}}$ and $T_{D_{min}}$. Which is meaning that for any giving distribution $P_d$ if we can find $D_max$ and $D_min$ we have:
$$\left| T_d-T_e \right|\leq\left| T_{D_{max}}-T_{D_{min}} \right|$$
\end{theorem}
\begin{IEEEproof}
It's obviously that we need make $P_R$, $P_L$, $P_E$ and the way to solve BGMDD problem keep the same when the duration is changed. In this situation, when the duration is bigger, the total score must be bigger. And Considered that $E_d$ is the expectation of  $P_d$, So we have the following three inequations:
{\setlength\abovedisplayskip{1pt}
\setlength\belowdisplayskip{1pt}
\begin{spacing}{1.0}
$$ T_{D_{min}}\leq T_d \leq T_{D_{max}} $$
$$D_{min}\leq E_d \leq D_{max}$$
$$ T_{D_{min}}\leq T_e \leq T_{D_{max}} $$
\par Make the first inequation subtract the third inequation, we have:
$$ T_{D_{min}}-T_{D_{max}}\leq T_d-T_e \leq T_{D_{max}}-T_{D_{min}} $$
\par Rearranging, we obtain that
$$\left| T_d-T_e \right|\leq\left| T_{D_{max}}-T_{D_{min}} \right|$$
\end{spacing}}
\end{IEEEproof}
\par As we don't consider the property of specific distribution, the bound is actually very loose. For some special dirtribution, like normal distribution, $D_{max}$ could be replaced by the value bigger then the 95\% values in distribution. And so as the $D_{min}$.
\par According to \textbf{Theorem 1},   the difference between $T_d$ and $T_e$ depends on the difference between $T_{D_{max}}$ and $T_{D_{min}}$. We  would prove that when the values of   $D_{max}$ and $D_{min}$ is much bigger then  the difference between $D_{max}$ and $D_{min}$ the difference between $T_{D_{max}}$ and $T_{D_{min}}$ would be small in section xxx. Under this situation, we can replace the distribution $P_d$ by its expection  $E_d$ to analyze the process.

\subsection{Batch-based method}
\par Batch-based method is to divide DBG probelm into a batch partition problem. The main idea of batch method is to wait for a specific duration(the length of batch) and then match the existing nodes in the graph by Hungarian algorithm[x]. It's a very simple and valid method to solve DBG problem, because when the node can wait for the better, batch method can accumulate enough nodes to make a better match.
\par It's obvious that the score of  a match at the end of  a batch duration depends on how much the nodes exsit in current graph. When the number of nodes is bigger, the score is bigger. 
\begin{theorem}\label{the score of  a match}
Assume $e_1, e_2,..., e_n$ are n independent identically distributed variables obeying $P_E$. Then we consider the expection of max value among $e_1, e_2,..., e_n$ as $X(n)$ which means $X(n)=E(max(e_1, e_2,..., e_n))$. When $P_E$ and $n$ are specified, $X(n)$ is determinated. And when $n$ is increasing, $X(n)$  is increasing which means the $X(n)$ is monotone increasing with $n$. When the number of left nodes is $L$ and the number of right number is $R$ at the matching moment, we have the bound of score expection:
$$\sum_{i=1}^n{X(m-i)} \leq E(T(L,R))\leq nX(m)$$ 
$$n=min(L,R), m=max(L,R)$$
where $T(L,R)$ means the total score of this matching when  the number of left nodes is $L$ and the number of right number is $R$.
\end{theorem}
\begin{IEEEproof}
 For convenience, we assume that $L$ is smaller than $R$. Then, we focus on left nodes, because there would be no left nodes left after matching when $L$ is smaller than $R$. In this situation, each left node has $R$ choices to match, but not all left node can choose the best choice because some best choices maybe correspond to the same right node. So the best situation is every left node's best choice is different from others, which means the expection of every left node's matching score is $X(R)$. So the total score is $L\times X(R)$ in the best situation. 
 \par Now, consider the worst situstion. We first choose the largest weight edge and match corresponding nodes. Next, find another left node and we find that its best choice is matched, so its choice just $R-1$ left. So the node need to choose from the $R-1$ choice. Then the next left node come across again, it's equivalent that the node just have $R-2$ choice to choose at first. The situation happens again and again. So the expection of total score in this situation is  $\sum_{i=1}^L{X(R-i)}$. 
 \par The worst situation could hardly happen if we use Hungarian algorithm to match the graph.xxxx So it's just the lower bound.xxxx
\end{IEEEproof}

\subsection{Optimal static batch duration at specific $E_d$}
\par The simplest batch based method to solve DBG problem is to match all exsiting nodes in the graph using Hungarian algorithm periodically. Intuitively, under a specific $E_d$, the length of matching period would affect the total score.
\par According to \emph{Theorem 2}, the total score in a matching depends on $L$ and $R$.  
\begin{algorithm}[H]
\caption{Put your caption here}
\begin{algorithmic}[1]

\Procedure{proc}{$a,b$}       \Comment{This is an example}
    \State System Initialization
    \State Read the value 
    \If{$condition = True$}
        \State Do this
        \If{$Condition \geq 1$}
        \State Do that
        \ElsIf{$Condition \neq 5$}
        \State Do another
        \State Do that as well
        \Else
        \State Do otherwise
        \EndIf
    \EndIf

    \While{$something \not= 0$}  \Comment{put some comments here}
        \State $var1 \leftarrow var2$  \Comment{another comment}
        \State $var3 \leftarrow var4$
    \EndWhile  \label{roy's loop}
\EndProcedure
\end{algorithmic}
\end{algorithm}



%!TEX root = main.tex
\section{evaluation}
\label{sec:eva}

Since evaluation section is where figures and tables appear the most, I put examples of inserting figures and tables here, but they can be used elsewhere.

\subsection{figures}

\begin{figure}[H]
\centering
\includegraphics[width=2.0in]{Fig/DCN.pdf}
\caption{insert one figure}
\label{fig:DCN}
\vspace{-3mm}
\end{figure}

\begin{figure}[H]
\subfigure[fig1]{
\begin{minipage}[b]{0.2\textwidth}
\label{fig:TopologyDCN}
\centering
\includegraphics[width=1.5in]{Fig/DCN.pdf}
\end{minipage}}
\subfigure[fig2]{
\begin{minipage}[b]{0.2\textwidth}
\label{fig:TopologyWAN}
\centering
\includegraphics[width=1.5in]{Fig/WAN.pdf}
\end{minipage}}
\caption{put two figures together horizontally} \label{fig:Topology}
\vspace{-1em}
\end{figure}


\begin{figure*}[t]
\begin{minipage}[t]{0.49\textwidth}
\subfigure[DCN scenario]{
\label{fig:DCN}
\includegraphics[width=1.6in]{Fig/DCN.pdf}}
\subfigure[WAN scenario]{
\label{fig:WAN}
\includegraphics[width=1.6in]{Fig/WAN.pdf}}
\caption{Maximum link congestion comparison.}
\label{fig:LinkCongestion}
\end{minipage}
\begin{minipage}[t]{0.49\textwidth}
\subfigure[DCN scenario]{
\label{fig:RunningTimeDCN}
\includegraphics[width=1.6in]{Fig/RTDCN.pdf}}
\subfigure[WAN scenario]{
\label{fig:RunningTimeWAN}
\includegraphics[width=1.6in]{Fig/RTWAN.pdf}}
\caption{The number of congested flows.}
\label{fig:RunningTime}
\end{minipage}
\vspace{-1em}
\end{figure*}

\subsection{tables}

\begin{table}[H]
\caption{Running time for finding congestion-free update plans}\label{Tab:RunningTime}
\centering
\begin{tabular}{|c|c|c|c|c|c|}
\hline
& 1K & 2K & 3K & 4K & 5K  \\
\hline
DCN & 0.73 min & 1.40 min & 2.10 min & 2.96 min &  4.12 min  \\
\hline
WAN & 0.60 min & 1.01 min & 1.57 min & 2.43 min & 3.12 min  \\
\hline
\end{tabular}
\begin{tabular}{l}
\\
write something to explain your table in here
\end{tabular}
\end{table}

%!TEX root = main.tex
\section{Conclusion}
\label{sec:conclusion}
In this paper, we list the basic component of an academic paper and show how they are organized using LaTex.

% remember \usepackage[cmex10]{amsmath}

% remember \usepackage{algorithmic} \usepackage{algorithm}


% remember  \usepackage[pdftex]{graphicx}  \usepackage[tight,footnotesize]{subfigure} \usepackage{tabularx}


%if figures don't align horizontally, try adjusting {0.x\textwidth} and [width=xinch]


\section{Acknowledgement}
We thank the anonymous reviewers for their helpful comments on draft of this paper.
The work is partly supported by XX project(maybe not).

\bibliography{cite}
\bibliographystyle{abbrv}

\input{appendices}

\end{document}